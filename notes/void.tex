\documentclass[10pt, a4paper, onecolumn, oneside, titlepage, openany]{book}

\usepackage[T1]{fontenc}    %% TeX text extended - most European characters
\usepackage[utf8]{inputenc} %% UTF-8 support
\usepackage{menukeys}       %% keyboads keys, menu + includes '\definecolor'

\usepackage{hyperref}       %% URL and reference support
\hypersetup{
    colorlinks=true,
    linkcolor=blue,         %% \ref{}
    filecolor=magenta,      %% \href{run:./file.txt}{File.txt}
    urlcolor=cyan,          %% \href{http://www.overleaf.com}{Link}
    pdfpagemode=FullScreen,
    pdftitle={Void}
}

\usepackage{titlesec}       %% chapter formatting (N. CHAPTER-NAME)
\renewcommand{\chaptername}{}
\titleformat{\chapter}[hang]{\normalfont\huge\bfseries}{\chaptertitlename\ \thechapter.}{1em}{}

\usepackage{fancyhdr}       %% decorative lines
\renewcommand{\headrulewidth}{2pt}
\renewcommand{\footrulewidth}{2pt}
\pagestyle{fancy}
\fancyhf{}
    \chead{\leftmark}
    \cfoot{\thepage}
\fancypagestyle{plain}{
\fancyhf{}
    \chead{\leftmark}
    \cfoot{\thepage}
}

%% table formatting
\usepackage[format=hang,font=small,labelfont=bf]{caption} %% bold table caption
\setlength{\arrayrulewidth}{0.5mm}                        %% border
\setlength{\tabcolsep}{18pt}                              %% space from border (X axis)
\renewcommand{\arraystretch}{1.5}                         %% space from border (Y axis)

%% colors
\definecolor{bg}{RGB}{240, 240, 240}
\definecolor{root}{RGB}{222, 0, 0}
\definecolor{user}{RGB}{0, 150, 0}
\definecolor{command}{RGB}{41, 182, 0}
\definecolor{block}{RGB}{255, 80, 0}
\definecolor{dir}{RGB}{0, 100, 200}
\definecolor{file}{RGB}{77, 187, 101}
\definecolor{comment}{RGB}{0, 182, 182}

\definecolor{keyword}{RGB}{204, 204, 0}

%% code formatting
\usepackage{fancyvrb}
\renewcommand{\FancyVerbFormatLine}[1]{\colorbox{bg}{#1}}

%% text file content formatting
\usepackage{listings}
%% Docs: https://texdoc.org/serve/listings.pdf/0 ; https://en.m.wikibooks.org/wiki/LaTeX/Source_Code_Listings ; % https://gordonlesti.com/custom-code-highlighting-in-latex/

\lstdefinelanguage{linux_file}{
    morekeywords = [1]{source,clear,set,umask,export},
    morekeywords = [2]{if,then,else,fi},
    sensitive = false,
    morecomment = [l]{\#},
    morestring = [b]",
    morestring = [b]',
}

\lstset{
    language = linux_file,
    basicstyle = \ttfamily,
    columns=fullflexible,
    keywordstyle = [1],%\color{command},
    keywordstyle = [2]\color{keyword},
    stringstyle = \color{block},
    commentstyle = \color{comment},
    keepspaces = true,
    numbers = left,
    firstnumber = 1,
    stepnumber = 1,
    numberstyle = \small\color{file},
    showspaces = false,
    showstringspaces = false,
    showtabs = false,
    breaklines = true,
    breakatwhitespace = false,
    breakindent=0pt,
    %postbreak=\space,
    %prebreak=\char92,
}

%%%%%%%%%%%%%%%%%%%%%%%%%%%%%%%%%%%%%%%%%%%%%%%%%%%%%%%%%%%%%%%%%%%%%%%%%%%%%%%%%%%%%%%%%%%%%%%%%%%%%%%%%%%%%%%%%

%% title page '\maketitle'
\title{\textbf{Void}}
\author{AISK11}
\date{May 2022}

%% LaTeX content
\begin{document}
\maketitle
\tableofcontents

\chapter{System Info}
\begin{table}[!ht]
\centering
\begin{tabular}{|c|c|}
    \hline
    \textbf{Hardware} & \textbf{Value} \\
    \hline
    CPU architecture & amd64 (x86\_64)\\
    CPU vendor & intel\\
    Motherboard & UEFI\\
    Hard Drive & NVMe\\
    Ethernet & r8169 (or r8168)\\
    WiFi & iwlwifi\\
    GPU & Intel + Nvidia\\
    \hline
\end{tabular}
\caption{System hardware overview.}
\label{table:system_hardware_overview}
\end{table}

\begin{table}[!ht]
\centering
\begin{tabular}{|c|c|}
    \hline
    \textbf{Void Linux} & \textbf{Value} \\
    \hline
    Architecutre & amd64 (x86\_64)\\
    C library & glibc\\
    ISO Image & x86\_64-glibc\\
    \hline
\end{tabular}
\caption{Void Linux overview.}
\label{table:void_linux_overview}
\end{table}

\begin{table}[!ht]
\centering
\begin{tabular}{|c|c|}
    \hline
    \textbf{Software} & \textbf{Value} \\
    \hline
    RAID & -\\
    LVM & -\\
    Encryption & LUKS\\
    Root filesystem & btrfs\\
    SWAP & 2 GB (file)\\
    Bootloader & rEFInd\\
    Kernel & ?\\
    Initramfs & ?\\
    Init & runit\\
    Shell & ?\\
    Virtualization & ?\\
    \hline
\end{tabular}
\caption{System software overview.}
\label{table:system_software_overview}
\end{table}

\begin{table}[!ht]
\centering
\begin{tabular}{|c|c|}
    \hline
    \textbf{Software} & \textbf{Value} \\
    \hline
    Display server & Xorg\\
    Window manager & ?\\
    Desktop environment & -\\
    Terminal emulator & ?\\
    \hline
\end{tabular}
\caption{Display software overview.}
\label{table:display_software_overview}
\end{table}



\chapter{Download And Boot}
\section{Download}
\begin{enumerate}
    \item \textbf{Download (ISO, checksum, signature and Void release key):}
    \begin{itemize}
        \item \textbf{Download website:}
        \begin{itemize}
            \item \url{https://voidlinux.org/download/}
        \end{itemize}
        \item \textbf{Latest image repo (image + checksum + signature):}
        \begin{itemize}
            \item \url{https://alpha.de.repo.voidlinux.org/live/current/}
        \end{itemize}
        \item \textbf{Mirrors:}
        \begin{itemize}
            \item \url{https://docs.voidlinux.org/xbps/repositories/mirrors/index.html}
        \end{itemize}
        \item \textbf{Void release keys:}
        \begin{itemize}
            \item \url{https://github.com/void-linux/void-packages/tree/master/srcpkgs/void-release-keys/files/}
        \end{itemize}
    \end{itemize}
    \item \textbf{Verify download:}
    \begin{enumerate}
        \item \textbf{Verify ISO integrity via checksum:}
\begin{Verbatim}[commandchars=\\\{\}]
\textcolor{user}{user\$} \textcolor{command}{sha256sum} -c --ignore-missing <\textcolor{file}{sha256sum.txt}>
<\textcolor{file}{void-live-x86_64-YYYYMMDD.iso}>
\end{Verbatim}
        \item \textbf{Verify checksum authenticity for ISO:}
\begin{Verbatim}[commandchars=\\\{\}]
\textcolor{user}{user\$} \textcolor{command}{signify} -V -p <\textcolor{file}{void-release-YYYYMMDD.pub}>
-x <\textcolor{file}{sha256sum.sig}> -m <\textcolor{file}{sha256sum.txt}>
\end{Verbatim}
    \end{enumerate}
\end{enumerate}


\section{Prepare USB:}
\begin{enumerate}
    \item \textbf{Unmount USB FS!}
    \item \textbf{Flash image to USB:}
\begin{Verbatim}[commandchars=\\\{\}]
\textcolor{root}{root#} \textcolor{command}{dd} if=<\textcolor{file}{void-live-x86_64-YYYYMMDD.iso}> of=<\textcolor{block}{/dev/sdX}>
[bs=4M] [status=progress] ; \textcolor{command}{sync}
\end{Verbatim}
\end{enumerate}


\section{Boot USB}
\subsection{Disable Secure Boot}
\begin{enumerate}
    \item \textbf{During POST press key to access BIOS/UEFI:}
\newline \href{https://techofide.com/blogs/boot-menu-option-keys-for-all-computers-and-laptops-updated-list-2021-techofide/}{https://techofide.com/blogs/boot-menu-option-keys-for-all-computers-and-laptops-updated-list-2021-techofide/}
    \item \textbf{Disable Secure Boot.}
    \item \textbf{Poweroff/Restart.}
\end{enumerate}
\subsection{Boot}
\begin{enumerate}
    \item \textbf{Plug in flashed USB.}
    \item \textbf{During POST press key to access Boot Menu:}
\newline \href{https://techofide.com/blogs/boot-menu-option-keys-for-all-computers-and-laptops-updated-list-2021-techofide/}{https://techofide.com/blogs/boot-menu-option-keys-for-all-computers-and-laptops-updated-list-2021-techofide/}
    \item \textbf{Select USB entry (UEFI).}
\end{enumerate}



\chapter{Pre-Installation}
\section{Login}
\begin{enumerate}
    \item \textbf{Login to the system:}
\begin{Verbatim}[commandchars=\\\{\}]
\textcolor{root}{>} root
\textcolor{root}{>} voidlinux
\end{Verbatim}
\end{enumerate}


\section{Change shell}
\begin{enumerate}
    \item \textbf{Change shell to bash:}
\begin{Verbatim}[commandchars=\\\{\}]
\textcolor{root}{root#} \textcolor{command}{bash}
\end{Verbatim}
\end{enumerate}


\section{Disable pcspkr @CHECK}
\begin{enumerate}
    \item \textbf{See if pcspkr is active:}
\begin{Verbatim}[commandchars=\\\{\}]
\textcolor{root}{root#} \textcolor{command}{lsmod} | \textcolor{command}{grep} -i pcspkr
\end{Verbatim}
    \item \textbf{Disable pcspkr module:}
\begin{Verbatim}[commandchars=\\\{\}]
\textcolor{root}{root#} \textcolor{command}{modprobe} -r pcspkr
\end{Verbatim}
\end{enumerate}


\section{Verify UEFI}
\begin{itemize}
    \item \textbf{UEFI is used if following directory exists:}
\newline \textbf{\textcolor{dir}{/sys/firmware/efi/}}
    \item \textbf{Verify via kernel:}
\begin{Verbatim}[commandchars=\\\{\}]
\textcolor{root}{root#} \textcolor{command}{dmesg} | \textcolor{command}{grep} -i efi
\end{Verbatim}
\end{itemize}


\section{Configure Internet @TODO}


\section{Install Additional Software @TODO}


\section{Disks}
\subsection{Theory}
\begin{itemize}
    \item \textbf{Sector:} smallest unit size on disk. 512 or 4096 bytes (Advanced Format). 4096 Advanced Format disks have usually 512-byte conversion firmware.
    \item \textbf{Block:} allocation size the FS uses. Cannot be smaller than size of the sector. Can be group of sectors (4096b: 8 x 512b sectors).
    \begin{itemize}
        \item \textbf{512b =} good for lot of small files. More blocks = more metadata.
        \item \textbf{4096b =}  good for larger files, less metadata. Waste if there are mostly small files.
    \end{itemize}
\end{itemize}

\subsection{Disk Information}
\begin{itemize}
    \item \textbf{Find disks (block devices):}
\begin{Verbatim}[commandchars=\\\{\}]
\textcolor{user}{user\$} \textcolor{command}{lsblk} [-a] [-p] [-f]
\textcolor{root}{root#} \textcolor{command}{fdisk -l} [\textcolor{block}{/dev/sdX}]
\textcolor{root}{root#} \textcolor{command}{blkid}
\end{Verbatim}
    \item \textbf{Get raw disk info:}
    \begin{itemize}
        \item \textbf{Get disk physical sector size:}
\begin{Verbatim}[commandchars=\\\{\}]
\textcolor{root}{root#} \textcolor{command}{blockdev} [-v] --getpbsz <\textcolor{block}{/dev/sdX[Y]}>
\end{Verbatim}
        \item \textbf{Get disk logical sector size (usually 512):}
\begin{Verbatim}[commandchars=\\\{\}]
\textcolor{root}{root#} \textcolor{command}{blockdev} [-v] --getss <\textcolor{block}{/dev/sdX[Y]}>
\end{Verbatim}
        \item \textbf{Disk size in bytes:}
\begin{Verbatim}[commandchars=\\\{\}]
\textcolor{root}{root#} \textcolor{command}{blockdev} [-v] --getsize64 <\textcolor{block}{/dev/sdX[Y]}>
\end{Verbatim}
    \item \textbf{Check if disk is readonly (1 = ro, 0 = rw):}
\begin{Verbatim}[commandchars=\\\{\}]
\textcolor{root}{root#} \textcolor{command}{blockdev} [-v] --getro <\textcolor{block}{/dev/sdX[Y]}>
\end{Verbatim}
    \end{itemize}
    \item \textbf{See partitions:}
\begin{Verbatim}[commandchars=\\\{\}]
\textcolor{root}{root#} \textcolor{command}{parted} -s <\textcolor{block}{/dev/sdX}> (p)rint [free]
\end{Verbatim}
\end{itemize}

\subsection{Bad Sectors Check}
\begin{enumerate}
    \item \textbf{Unmount FS!}
    \item \textbf{Check disk for bad blocks:}
\begin{Verbatim}[commandchars=\\\{\}]
\textcolor{root}{root#} \textcolor{command}{badblocks} [-b 4096] [-w [-t 0xaa]] [-v] [-s]
<\textcolor{block}{/dev/sdX[Y]}> [-o <\textcolor{file}{OUTPUT.txt}>]
\end{Verbatim}
\end{enumerate}



\chapter{Installation}
\section{Disk Preparation}
\subsection{Disk Partitioning}
\begin{enumerate}
    \item \textbf{Umount every FS from the disk that is going to be partitioned!}
    \item \textbf{Create partitions:}
    \begin{enumerate}
        \item \textbf{Enter cfdisk:}
\begin{Verbatim}[commandchars=\\\{\}]
\textcolor{root}{root#} \textcolor{command}{cfdisk} -z <\textcolor{block}{/dev/sdX}>
\end{Verbatim}
        \item \textbf{Create partition table:}
\begin{Verbatim}[commandchars=\\\{\}]
\textcolor{root}{cfdisk>} \textcolor{command}{gpt} (Enter)
\end{Verbatim}
        \item \textbf{Create EFI partition:}
\begin{Verbatim}[commandchars=\\\{\}]
\textcolor{root}{cfdisk>} \textcolor{command}{n}
\textcolor{root}{cfdisk>} \textcolor{command}{550MiB}
\textcolor{root}{cfdisk>} \textcolor{command}{t}
\textcolor{root}{cfdisk>} \textcolor{command}{EFI System}
\end{Verbatim}
        \item \textbf{Create LUKS partition:}
\begin{Verbatim}[commandchars=\\\{\}]
\textcolor{root}{cfdisk>} \textcolor{command}{n}
\textcolor{root}{cfdisk>} \textcolor{command}{} (Enter)
\end{Verbatim}
        \item \textbf{Write changes:}
\begin{Verbatim}[commandchars=\\\{\}]
\textcolor{root}{cfdisk>} \textcolor{command}{W}
\textcolor{root}{cfdisk>} \textcolor{command}{yes}
\end{Verbatim}
        \item \textbf{Quit cfdisk:}
\begin{Verbatim}[commandchars=\\\{\}]
\textcolor{root}{cfdisk>} \textcolor{command}{Q}
\end{Verbatim}
        \item \textbf{Name partitions:}
\begin{Verbatim}[commandchars=\\\{\}]
\textcolor{root}{root#} \textcolor{command}{parted} -s <\textcolor{block}{/dev/sdX}> name 1 ESP
\textcolor{root}{root#} \textcolor{command}{parted} -s <\textcolor{block}{/dev/sdX}> name 2 LUKS
\end{Verbatim}
    \end{enumerate}
    \item \textbf{Create filesystems:}
    \begin{enumerate}
        \item \textbf{ESP (VFAT32):}
\begin{Verbatim}[commandchars=\\\{\}]
\textcolor{root}{root#} \textcolor{command}{wipefs} -a <\textcolor{block}{/dev/sdX1}>
\textcolor{root}{root#} \textcolor{command}{mkfs.vfat} -F 32 <\textcolor{block}{/dev/sdX1}>
\textcolor{root}{root#} \textcolor{command}{fatlabel} <\textcolor{block}{/dev/sdX1}> ESP
\textcolor{root}{root#} \textcolor{command}{fatlabel} -i <\textcolor{block}{/dev/sdX1}> 00000001
\textcolor{root}{root#} \textcolor{command}{fsck.vfat} -v <\textcolor{block}{/dev/sdX1}> ; \textcolor{command}{echo} \$?
\end{Verbatim}
        \item \textbf{LUKS:}
\begin{Verbatim}[commandchars=\\\{\}]
\textcolor{root}{root#} \textcolor{command}{wipefs} -a <\textcolor{block}{/dev/sdX2}>
\textcolor{root}{root#} \textcolor{command}{cryptsetup} luksFormat -y --type luks2 <\textcolor{block}{/dev/sdX2}>
\textcolor{root}{>} YES
\textcolor{root}{>} <NEW_LUKS_PASSWORD>
\textcolor{root}{>} <NEW_LUKS_PASSWORD (VERIFY)>
\textcolor{root}{root#} \textcolor{command}{cryptsetup} config --label LUKS <\textcolor{block}{/dev/sdX2}>
\textcolor{root}{root#} \textcolor{command}{cryptsetup} luksUUID --uuid
00000000-0000-0000-0000-000000000002 <\textcolor{block}{/dev/sdX2}>
\textcolor{root}{>} YES
\textcolor{root}{root#} \textcolor{command}{cryptsetup} open --type luks2 <\textcolor{block}{/dev/sdX2}>
luks-root
\textcolor{root}{>} <PASSWORD>
\end{Verbatim}
        \item \textbf{Root FS (btrfs):}
\begin{Verbatim}[commandchars=\\\{\}]
\textcolor{root}{root#} \textcolor{command}{wipefs} -a \textcolor{block}{/dev/mapper/luks-root}
\textcolor{root}{root#} \textcolor{command}{mkfs.btrfs} \textcolor{block}{/dev/mapper/luks-root}
\textcolor{root}{root#} \textcolor{command}{btrfs} filesystem label \textcolor{block}{/dev/mapper/luks-root}
LUKS-ROOT
\textcolor{root}{root#} \textcolor{command}{btrfstune} -U 00000000-0000-0000-0000-000000000003
\textcolor{block}{/dev/mapper/luks-root}
\textcolor{root}{>} y
\textcolor{root}{root#} \textcolor{command}{btrfs} check -p \textcolor{block}{/dev/mapper/luks-root} ; \textcolor{command}{echo} \$?
\end{Verbatim}
    \end{enumerate}
    \item \textbf{Finish and clean:}
\begin{Verbatim}[commandchars=\\\{\}]
\textcolor{root}{root#} \textcolor{command}{cryptsetup} close \textcolor{block}{/dev/mapper/luks-root}
\end{Verbatim}
\end{enumerate}

\subsection{LUKS Cheatsheet:}
\begin{itemize}
    \item \textbf{Open LUKS:}
\begin{Verbatim}[commandchars=\\\{\}]
\textcolor{root}{root#} \textcolor{command}{cryptsetup} open --type luks <\textcolor{block}{/dev/sdX2}> <luks>
> <PASSWORD>
\end{Verbatim}
    \item \textbf{Close LUKS:}
\begin{Verbatim}[commandchars=\\\{\}]
\textcolor{root}{root#} \textcolor{command}{cryptsetup} close <\textcolor{block}{/dev/mapper/luks}>
\end{Verbatim}
    \item \textbf{LUKS header:}
    \begin{enumerate}
        \item \textbf{See LUKS header:}
\begin{Verbatim}[commandchars=\\\{\}]
\textcolor{root}{root#} \textcolor{command}{cryptsetup} luksDump <\textcolor{block}{/dev/sdX2}>
\end{Verbatim}
        \item \textbf{Make LUKS header backup:}
\begin{Verbatim}[commandchars=\\\{\}]
\textcolor{root}{root#} \textcolor{command}{cryptsetup} luksHeaderBackup <\textcolor{block}{/dev/sdX2}>
--header-backup-file <\textcolor{file}{FILE}>
\end{Verbatim}
        \item \textbf{Destroy LUKS header:}
\begin{Verbatim}[commandchars=\\\{\}]
\textcolor{root}{root#} \textcolor{command}{cryptsetup} luksErase <\textcolor{block}{/dev/sdX2}>
> YES
\end{Verbatim}
        \item \textbf{Restore LUKS header:}
\begin{Verbatim}[commandchars=\\\{\}]
\textcolor{root}{root#} \textcolor{command}{cryptsetup} luksHeaderRestore <\textcolor{block}{/dev/sdX2}>
--header-backup-file <\textcolor{file}{FILE}>
> YES
\end{Verbatim}
    \end{enumerate}
    \item \textbf{Passwords}
    \begin{itemize}
        \item \textbf{Change password:}
\begin{Verbatim}[commandchars=\\\{\}]
\textcolor{root}{root#} \textcolor{command}{cryptsetup} luksChangeKey <\textcolor{block}{/dev/sdX2}> [-S 0]
> <OLD_PASSWORD>
> <NEW_PASSWORD>
> <NEW_PASSWORD (VERIFY)>
\end{Verbatim}
        \item \textbf{Add another key:}
\begin{Verbatim}[commandchars=\\\{\}]
\textcolor{root}{root#} \textcolor{command}{cryptsetup} luksAddKey <\textcolor{block}{/dev/sdX2}> [-S 7]
> <ANY_KEY_PASSWORD>
> <NEW_KEY_PASSWORD>
> <NEWKEY_PASSWORD (VERIFY)>
\end{Verbatim}
        \item \textbf{Delete key:}
\begin{Verbatim}[commandchars=\\\{\}]
\textcolor{root}{root#} \textcolor{command}{cryptsetup} luksKillSlot <\textcolor{block}{/dev/sdX2}> [7]
> <ANY_KEY_PASSWORD>
\end{Verbatim}
    \end{itemize}
\end{itemize}

\end{document}
